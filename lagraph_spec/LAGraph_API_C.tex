\documentclass[11pt]{extbook}

\usepackage{geometry}

\usepackage{fancyvrb}
\usepackage{color}
\usepackage{graphicx}
\usepackage{fullpage}
\usepackage{verbatim}
\usepackage{tikz}
\usepackage{listings}
\usepackage[yyyymmdd,hhmmss]{datetime}
\usepackage{rotating}
\usepackage{authblk}
\usepackage{amsfonts}
\usepackage{amsmath}
\usepackage{amssymb}
\usepackage{todonotes}
\usepackage{titlesec}
\usepackage[mathlines]{lineno}
\usepackage{tabularx}
\usepackage{enumitem}
\usepackage{hyperref}
\usepackage{bm}
\usepackage{etoolbox}
\usepackage{pdflscape}
\usepackage{threeparttable}

%TGM:  Added these packages to fix underscore rendering
\usepackage{lmodern} 
\usepackage[T1]{fontenc}

\setcounter{secnumdepth}{3}
\setcounter{tocdepth}{3}

%\usepackage{draftwatermark}
%\SetWatermarkText{DRAFT}
%\SetWatermarkScale{2}

\renewcommand{\thefootnote}{\fnsymbol{footnote}}
\setcounter{footnote}{1}

\titleformat{\paragraph}
{\normalfont\normalsize\bfseries}{\theparagraph}{1em}{}
\titlespacing*{\paragraph}
{0pt}{3.25ex plus 1ex minus .2ex}{1.5ex plus .2ex}

\newtoggle{assign}
\toggletrue{assign}

\newcommand{\qg}{\u{g}}
\newcommand{\qG}{\u{G}}
\newcommand{\qc}{\c{c} }
\newcommand{\qC}{\c{C}}
\newcommand{\qs}{\c{s}}
\newcommand{\qS}{\c{S}}
\newcommand{\qu}{\"{u}}
\newcommand{\qU}{\"{U}}
\newcommand{\qo}{\"{o}}
\newcommand{\qO}{\"{O}}
\newcommand{\qI}{\.{I}}
\newcommand{\wa}{\^{a}}
\newcommand{\wA}{\^{A}}

\begin{document}

\linenumbers

\title{
The LAGraph C API Specification
{\large Version 0.5} \\
}

%%
%% I want this list to include people who actively work on the the text of the specification.  We have
%% an acknowledgments section for people who work with us on a more casual basis.
%%
\author{David Bader, Tim Davis, Jim Kitchen, Scott Kolodziej, Tim Mattson, Scott McMillan, 
and others as they take on writing tasks}
%  I wanted to safe his name so I could get it write later if we add him..... Ayd\i n Bulu\c{c}, 
\date{Generated on \today\ at \currenttime\ PDT}

\newcommand{\kron}{\mathbin{\text{\footnotesize \textcircled{\raisebox{-0.3pt}{\footnotesize $\otimes$}}}}}
\newcommand{\grbarray}[1]{\bm{#1}}
\renewcommand{\vector}[1]{{\bf #1}}
\renewcommand{\matrix}[1]{{\bf #1}}
\renewcommand{\arg}[1]{{\sf #1}}
\newcommand{\zip}{{\mbox{zip}}}
\newcommand{\zap}{{\mbox{zap}}}
\newcommand{\ewiseadd}{{\mbox{\bf ewiseadd}}}
\newcommand{\ewisemult}{{\mbox{\bf ewisemult}}}
\newcommand{\mxm}{{\mbox{\bf mxm}}}
\newcommand{\vxm}{{\mbox{\bf vxm}}}
\newcommand{\mxv}{{\mbox{\bf mxv}}}
\newcommand{\gpit}[1]{{\sf #1}}
\newcommand{\ie}{{i.e.}}
\newcommand{\eg}{{e.g.}}
\newcommand{\nan}{{\sf NaN}}
\newcommand{\nil}{{\bf nil}}
\newcommand{\ifif}{{\bf if}}
\newcommand{\ifthen}{{\bf then}}
\newcommand{\ifelse}{{\bf else}}
\newcommand{\ifendif}{{\bf endif}}
\newcommand{\zero}{{\bf 0}}
\newcommand{\one}{{\bf 1}}
\newcommand{\true}{{\sf true}}
\newcommand{\false}{{\sf false}}
\newcommand{\syntax}{{C Syntax}}

\newcommand{\Dinn}{\mbox{$D_{in}$}}
\newcommand{\Din}[1]{\mbox{$D_{in_{#1}}$}}
\newcommand{\Dout}{\mbox{$D_{out}$}}

\newcommand{\bDinn}{\mbox{$\mathbf{D}_{in}$}}
\newcommand{\bDin}[1]{\mbox{$\mathbf{D}_{in_{#1}}$}}
\newcommand{\bDout}{\mbox{$\mathbf{D}_{out}$}}

%\newcommand{\aydin}[1]{{{\color{orange}[Aydin: #1]}}}
%\newcommand{\scott}[1]{{{\color{violet}[Scott: #1]}}}
%\newcommand{\tim}[1]{{{\color{teal}[Tim: #1]}}}
%\newcommand{\jose}[1]{{{\color{red}[Jose: #1]}}}
%\newcommand{\carl}[1]{{{\color{blue}[Carl: #1]}}}
%\newcommand{\ajy}[1]{{{\color{brown}[Yzelman: #1]}}}

% These commands are used to insert comments readible in the rendered text.
\newcommand{\aydin}[1]{}
\newcommand{\scott}[1]{}
\newcommand{\tim}[1]{}
\newcommand{\jose}[1]{}

%\carl{testing}
%\scott{testing}
%\aydin{testing}
%\tim{testing}
%\jose{testing}
%\ajy{testing}

\renewcommand{\comment}[1]{{}}
\newcommand{\glossBegin}{\begin{itemize}}
\newcommand{\glossItem}[1]{\item \emph{#1}: }
\newcommand{\glossEnd}{\end{itemize}}

\setlength{\parskip}{0.5\baselineskip}
\setlength{\parindent}{0ex}

\maketitle


\renewcommand{\thefootnote}{\arabic{footnote}}
\setcounter{footnote}{0}

\vfill

%% update copyright for all institutions authors are associated with

Copyright \copyright\ 2020-2021 Carnegie Mellon University, 
Texas A\&M University, and Intel Corporation.  %and more as official authors join in 

Any opinions, findings and conclusions or recommendations expressed in 
this material are those of the author(s) and do not necessarily reflect 
the views of the United States Department of Defense, the United States 
Department of Energy, Carnegie Mellon University, Texas A\&M University, or the Intel Corporation.  

NO WARRANTY. THIS MATERIAL IS FURNISHED ON AN AS-IS BASIS. THE COPYRIGHT 
OWNERS AND/OR AUTHORS MAKE NO WARRANTIES OF ANY KIND, EITHER EXPRESSED 
OR IMPLIED, AS TO ANY MATTER INCLUDING, BUT NOT LIMITED TO, WARRANTY OF 
FITNESS FOR PURPOSE OR MERCHANTABILITY, EXCLUSIVITY, OR RESULTS OBTAINED 
FROM USE OF THE MATERIAL. THE COPYRIGHT OWNERS AND/OR AUTHORS DO NOT MAKE 
ANY WARRANTY OF ANY KIND WITH RESPECT TO FREEDOM FROM PATENT, TRADE MARK, 
OR COPYRIGHT INFRINGEMENT.

\vspace{1.5cm}

\vspace{2cm}

\vspace{1.5cm}


Except as otherwise noted, this material is licensed under a Creative Commons Attribution 4.0 license (\href{http://creativecommons.org/licenses/by/4.0/legalcode}{http://creativecommons.org/licenses/by/4.0/legalcode}), 
and examples are licensed under the BSD License (\href{https://opensource.org/licenses/BSD-3-Clause}{https://opensource.org/licenses/BSD-3-Clause}).

%\begin{abstract}
%\end{abstract}

\vfill

\pagebreak
\tableofcontents
\vfill
\pagebreak

%-----------------------------------------------------------------------------
\addcontentsline{toc}{section}{List of Tables}
\listoftables
\vfill
\pagebreak

\addcontentsline{toc}{section}{List of Figures}
\listoffigures
\vfill
\pagebreak

%-----------------------------------------------------------------------------

\section*{Acknowledgments}
\addcontentsline{toc}{section}{Acknowledgments}

This document represents the work of the people who have served on the C API
LAGraph subcommittee of the GraphBLAS Forum.

Those who served as C API subcommittee members for LAGraph 0.5 are (in alphabetical order):
\begin{itemize}
\item David Bader (New Jersey Institute of Technology)
\item Ayd\i n Bulu\c{c} (Lawrence Berkeley National Laboratory)
\item Tim Davis (Texas A\&M)
\item James Kitchen (Anaconda)
\item Scott Kolodziej (Texas A\&M)
\item Timothy G. Mattson (Intel Corporation)
\item Scott McMillan (Software Engineering Institute at Carnegie Mellon University)
\item Jos\'e Moreira (IBM Corporation)
\item Gabor  Szarnyas (???)
\end{itemize}



The following people provided valuable input and feedback during the development of the specification (in alphabetical order):
\emph{to be added later}.
\vfill
\pagebreak

%-----------------------------------------------------------------------------

\chapter{Introduction}

The LAGraph C API defines a library of graph algorithms based on a representation
of graphs in terms of linear algebra~\cite{KepGil2011}.  We assume an implementation of the LAGraph library
is layered on top of a GraphBLAS library~\cite{buluc2019graphblas} and the objects manipulated by LAGraph are
constructed from GraphBLAS objects.  


LAGraph is built on a collection of   
objects exposed to the C programmer as opaque data types. 
Functions that manipulate these
objects are referred to as {\it methods}.  These methods fully define the 
interface to LAGraph objects to create or destroy them, modify their 
contents, and copy the contents of opaque objects into non-opaque objects; the 
contents of which are under direct control of the programmer.

%
%TGM:  I copied this text from the GraphBLAS spec.  We need to decide which
% version of C we require
The LAGraph C API is designed to work with C99 (ISO/IEC 9899:199) 
extended with {\it static type-based} and {\it number of parameters-based}
function polymorphism, and language extensions on par with 
the {\tt \_Generic} construct from C11 (ISO/IEC 9899:2011).  
%
%TGM:  Do we need to require this?
%Furthermore, the standard assumes programs using the GraphBLAS C API
%will execute on hardware that supports floating point arithmetic
%such as that defined by the IEEE~754 (IEEE 754-2008) standard. 

The remainder of this document is organized as follows:
\begin{itemize}
\item Chapter~\ref{Chp:Concepts}: Basic Concepts
\item Chapter~\ref{Chp:Objects}: Objects
\item Chapter~\ref{Chp:Functions}: Functions
\item Appendix~\ref{Chp:RevHistory}: Revision History
\item Appendix~\ref{Chp:Examples}: Examples
\end{itemize}

%=============================================================================
%=============================================================================

\chapter{Basic Concepts}
\label{Chp:Concepts}

The GraphBLAS C API is used to construct  
graph algorithms expressed ``in the language of linear algebra.''
Graphs are expressed as matrices, and the operations over 
these matrices are generalized through the use of a
semiring algebraic structure.

In this chapter, we will define the basic concepts used to
define the GraphBLAS C API.  We provide the following elements:
\begin{itemize}
\item Glossary of terms used in this document.  

\item Notation

\item Execution model

\item Error model

\end{itemize}

\section{Glossary}

%TGM I'm leaving a few definitions in here just as examples

\subsection{Basic definitions}

\glossBegin

\glossItem{application} A program that calls methods from the GraphBLAS C API to
solve a problem.

\glossItem{GraphBLAS C API} The application programming interface that fully defines the types, objects, 
literals, and other elements of the C binding to the GraphBLAS.

\glossEnd

\subsection{Objects and their structure}

\glossBegin
\glossItem{handle}  A variable that uses one of the GraphBLAS opaque data types.
The value of this variable holds a reference to a GraphBLAS object but not the contents of the object itself.
Hence, assigning a value of one handle to another variable copies the reference to the GraphBLAS object
but not the contents of the object.

\glossItem{non-opaque datatype} Any datatype that exposes its internal structure.   
This is contrasted
with an \emph{opaque datatype} that hides its internal structure and can
be manipulated only through an API.

\glossEnd



\vfill

\newgeometry{left=2.5cm,top=2cm,bottom=2cm}

\section{Notation}

\begin{tabular}[H]{l|p{5in}}
Notation & Description \\
\hline
$\Dout, \Dinn, \Din1, \Din2$  & Refers to output and input domains of various GraphBLAS operators. \\
$\bDout(*), \bDinn(*),$ & Evaluates to output and input domains of GraphBLAS operators (usually \\
~~~~$\bDin1(*), \bDin2(*)$ & a unary or binary operator, or semiring). \\
$\mathbf{D}(*)$   & Evaluates to the (only) domain of a GraphBLAS object (usually a monoid, vector, or matrix). \\ 
$f$             & An arbitrary unary function, usually a component of a unary operator. \\
$\mathbf{f}(F_u)$ & Evaluates to the unary function contained in the unary operator given as the argument. \\
$\odot$         & An arbitrary binary function, usually a component of a binary operator. \\
$\mathbf{\bigodot}(*)$ & Evaluates to the binary function contained in the binary operator or monoid given as the argument. \\
$\otimes$       & Multiplicative binary operator of a semiring. \\
$\oplus$        & Additive binary operator of a semiring. \\
$\mathbf{\bigotimes}(S)$ & Evaluates to the multiplicative binary operator of the semiring given as the argument. \\
$\mathbf{\bigoplus}(S)$ & Evaluates to the additive binary operator of the semiring given as the argument. \\
$\mathbf{0}(*)$   & The identity of a monoid, or the additive identity of a GraphBLAS semiring. \\
$\mathbf{L}(*)$   & The contents (all stored values) of the vector or matrix GraphBLAS objects.  For a vector, it is the set of (index, value) pairs, and for a matrix it is the set of (row, col, value) triples. \\
$\mathbf{v}(i)$ or $v_i$   & The $i^{th}$ element of the vector $\vector{v}$.\\
$\mathbf{size}(\vector{v})$ & The size of the vector $\vector{v}$.\\
$\mathbf{ind}(\vector{v})$ & The set of indices corresponding to the stored values of the vector $\vector{v}$.\\
$\mathbf{nrows}(\vector{A})$ & The number of rows in the $\matrix{A}$.\\
$\mathbf{ncols}(\vector{A})$ & The number of columns in the $\matrix{A}$.\\
$\mathbf{indrow}(\vector{A})$ & The set of row indices corresponding to rows in $\matrix{A}$ that have stored values.  \\
$\mathbf{indcol}(\vector{A})$ & The set of column indices corresponding to columns in $\matrix{A}$ that have stored values. \\
$\mathbf{ind}(\vector{A})$ & The set of $(i,j)$ indices corresponding to the stored values of the matrix. \\
$\mathbf{A}(i,j)$ or $A_{ij}$ & The element of $\matrix{A}$ with row index $i$ and column index $j$.\\
$\matrix{A}(:,j)$ & The $j^{th}$ column of the the matrix $\matrix{A}$.\\
$\matrix{A}(i,:)$ & The $i^{th}$ row of the the matrix $\matrix{A}$.\\
$\matrix{A}^T$ &The transpose of the matrix $\matrix{A}$. \\
$\neg\matrix{M}$ & The complement of $\matrix{M}$.\\
$\vector{\widetilde{t}}$ & A temporary object created  by the GraphBLAS implementation. \\
$<type>$ & A method argument type that is {\sf void *} or one of the types from Table~\ref{Tab:PredefinedTypes}. \\
{\sf GrB\_ALL} & A method argument literal to indicate that all indices of an input array should be used.\\
{\sf GrB\_Type} & A method argument type that is either a user defined type or one of the  types from Table~\ref{Tab:PredefinedTypes}.\\
{\sf GrB\_Object} &  A method argument type referencing any of the GraphBLAS object types.\\
{\sf GrB\_NULL} & The GraphBLAS NULL.\\
\end{tabular}

\restoregeometry

\section{Error model}



\section{Execution Model}
\label{Sec:ExecutionModel}

%% TGM: I've left the GraphBLAS model here just as a point of reference.

A program using the GraphBLAS C API constructs GraphBLAS objects,
manipulates them to implement a graph algorithm, and then extracts
values from the GraphBLAS objects as the result of the algorithm.
Functions defined within the GraphBLAS C API that manipulate GraphBLAS
objects are called \emph{methods}.  If the method corresponds to one
of the operations defined in the GraphBLAS mathematical specification,
we refer to the method as an \emph{operation}.

Graph algorithms are expressed as an ordered collection of GraphBLAS
method calls defined by the order they are encountered in a program.
This is called the \emph{program order}.  Each method in the collection
uniquely and unambiguously defines the output GraphBLAS objects based
on the GraphBLAS operation and the input GraphBLAS objects. This is the
case as long as there are no execution errors, which can put objects in
an invalid state (see Section~\ref{Sec:ErrorModel}).

The GraphBLAS method calls in program order are organized into contiguous
and nonoverlapping \emph{sequences}.  A sequence is an ordered collection
of method calls as encountered by an executing thread. (For more on
threads and GraphBLAS, see Section~\ref{Sec:ThreadSafety}.)  A sequence
begins with either (1) the first GraphBLAS method called by a thread,
or (2) the first method called by a thread after the end of the
previous sequence.  A sequence can end (terminate) in a variety of ways.
A call to the GraphBLAS {\sf GrB\_wait()} method (Section~\ref{Sec:GrB_wait})
always ends a sequence.  The GraphBLAS {\sf GrB\_finalize()} method
(Section~\ref{Sec:GrB_finalize}) also implicitly ends a sequence. Finally,
in blocking mode (see below), each GraphBLAS method starts and ends its
own sequence.

The GraphBLAS objects are fully defined at any point in a sequence by
the methods in the sequence as long as there are no execution errors.
In particular, as soon as a GraphBLAS method call returns, its output
can be used in the next GraphBLAS method call.  However, individual
operations in a sequence may not be \emph{complete}. We say that an
operation is complete when all the computations in the operation have
finished and all the values of its output object have been produced and
committed to the address space of the program. Furthermore, no additional
execution time can be charged to a completed operation and no additional
errors can be attributed to a completed operation.

The opaqueness of GraphBLAS objects allows execution to proceed
from one method to the next even when operations are not complete.
Processing of nonopaque objects is never deferred in GraphBLAS. That is,
methods that consume nonopaque objects (\eg, {\sf GrB\_Matrix\_build()},
Section~\ref{Sec:Matrix_build}) and methods that produce nonopaque objects (\eg,
{\sf GrB\_Matrix\_extractTuples()}, Section~\ref{Sec:Matrix_extractTuples})
always finish consuming or producing those nonopaque objects before
returning.   

\comment{Furthermore, methods that extract values from opaque GraphBLAS objects
into nonopaque user objects (see Table~\ref{Tab:ExtractMethods})
always force completion of all pending computations on the 
corresponding GraphBLAS source object.

\begin{table}[htb]
    \hrule
    \begin{center}
        \caption{Methods that extract values from a GraphBLAS object, thereby
        forcing completion of the operations contributing to that particular object.}
        \label{Tab:ExtractMethods}

        \begin{tabular}{l|l}
            Method    & Section \\ \hline

            {\sf GrB\_Vector\_nvals}        & \ref{Sec:Vector_nvals}        \\
            {\sf GrB\_Vector\_extractElement}     & \ref{Sec:extract_single_element_vec}    \\
            {\sf GrB\_Vector\_extractTuples}    & \ref{Sec:Vector_extractTuples}    \\
            {\sf GrB\_Matrix\_nvals}        & \ref{Sec:Matrix_nvals}        \\
            {\sf GrB\_Matrix\_extractElement}     & \ref{Sec:extract_single_element_mat}    \\
            {\sf GrB\_Matrix\_extractTuples}    & \ref{Sec:Matrix_extractTuples}    \\
            {\sf GrB\_reduce} (vector-scalar variant)        & \ref{Sec:Reduce_vector_scalar}        \\
            {\sf GrB\_reduce} (matrix-scalar variant)        & \ref{Sec:Reduce_matrix_scalar}        \\
        \end{tabular}
    \end{center}
    \hrule
\end{table}
}


\section{Error Model}
\label{Sec:ErrorModel}

%%TGM  I've left the GraphBLAS model here as a point of reference

All GraphBLAS methods return a value of type {\sf GrB\_Info} to provide
information available to the system at the time the method returns. The
returned value can be either {\sf GrB\_SUCCESS} or one of the defined
error values shown in Table~\ref{Tab:ErrorValues}. The errors fall into
two groups: API errors (Table~\ref{Tab:ErrorValues}(a)) and execution
errors (Table~\ref{Tab:ErrorValues}(b)).

\begin{table}[bh]
\hrule
\begin{center}
\caption{Error values returned by GraphBLAS methods.}
\label{Tab:ErrorValues}

\vspace{1\baselineskip}
(a) API errors
\vspace{1\baselineskip}

\begin{tabular}{l|p{3in}}
Error code    & Description \\ \hline
{\sf GrB\_UNINITIALIZED\_OBJECT} & A GraphBLAS object is passed to a method before {\sf new} was called on it.\\
{\sf GrB\_NULL\_POINTER} & A NULL is passed for a pointer parameter. \\
{\sf GrB\_INVALID\_VALUE} & Miscellaneous incorrect values. \\
{\sf GrB\_INVALID\_INDEX} & Indices passed are larger than dimensions of the matrix or vector being accessed. \\
{\sf GrB\_DOMAIN\_MISMATCH} & A mismatch between domains of collections and operations when user-defined domains are in use.\\
{\sf GrB\_DIMENSION\_MISMATCH} & Operations on matrices and vectors with incompatible dimensions. \\
{\sf GrB\_OUTPUT\_NOT\_EMPTY} & An attempt was made to build a matrix or vector using an output object that already contains valid tuples (elements).\\
%{\sf GrB\_NO\_VALUE} & An attempt was made to extract a value from a tuple within a matrix or vector for which there is no stored value. 
{\sf GrB\_NO\_VALUE} & A location in a matrix or vector is being accessed that has no stored value at the specified location. \scott{It depends on whether or not the non-opaque scalar is
well-defined on return from {\sf extract}}\\
\end{tabular}

\vspace{1\baselineskip}
(b) Execution errors
\vspace{1\baselineskip}

\begin{tabular}{l|p{3in}}
Error code    & Description \\ \hline
{\sf GrB\_OUT\_OF\_MEMORY}         & Not enough memory for operations. \\
{\sf GrB\_INSUFFICIENT\_SPACE}     & The array provided is not large enough to hold output. \\
{\sf GrB\_INVALID\_OBJECT}         & One of the opaque GraphBLAS objects (input or output) is in an invalid state caused by a previous execution error. \\
{\sf GrB\_INDEX\_OUT\_OF\_BOUNDS}  & Reference to a vector or matrix element that is outside the defined dimensions of the object. \\
{\sf GrB\_PANIC}        & Unknown internal error. \\
\end{tabular}

\end{center}
\hrule
\end{table}

An API error means that a GraphBLAS method was called with parameters that
violate the rules for that method.  These errors are restricted to those
that can be determined by inspecting the types and domains of GraphBLAS
objects, GraphBLAS operators, or the values of scalar parameters fixed at
the time a method is called.  API errors are deterministic and consistent
across platforms and implementations.  API errors are never deferred,
even in nonblocking mode. That is, if a method is called in a manner
that would generate an API error, it always returns with the appropriate
API error value.  If a GraphBLAS method returns with an API error, it
is guaranteed that none of the arguments to the method (or any other
program data) have been modified.

Execution errors indicate that something went wrong during the execution
of a legal GraphBLAS method invocation.  Their occurrence may depend on
specifics of the executing environment and data values being manipulated.
This does not mean that execution errors are the fault of the GraphBLAS
implementation.  For example, a memory leak could arise from an error in
an application's source code (a ``program error''), but it may manifest
itself in different points of a program's execution (or not at all)
depending on the platform, problem size, or what else is running at
that time.  Index-out-of-bounds and insuficient space execution errors
always indicate a program error.

In blocking mode, where each method executes to completion, a returned
execution error value applies to the specific method.  If a GraphBLAS
method, executing in blocking mode, returns with any execution error
from Table~\ref{Tab:ErrorValues}(b) other than {\sf GrB\_PANIC}, it
is guaranteed that no argument used as input-only has been modified.
Output arguments may be left in an invalid state, and their use downstream
in the program flow may cause additional errors.  If a GraphBLAS method
returns with a {\sf GrB\_PANIC} execution error, no guarantees can be
made about the state of any program data.

In nonblocking mode, execution errors can be deferred.  A return value
of {\sf GrB\_SUCCESS} only guarantees that there are no API errors in
the method invocation.  If an execution error value is returned by a
method in nonblocking mode, it indicates that an error was found during
execution of the sequence, up to and including the {\sf GrB\_wait()}
method (Section~\ref{Sec:GrB_wait}) call that ends the sequence. When possible, that return value
will provide information concerning the cause of the error.

As discussed in Section~\ref{Sec:GrB_waitOne}, a {\sf GrB\_wait(obj)} on
a specific GraphBLAS object {\sf obj} does not necessarily end a
sequence. However, no additional errors on the methods of the sequence that 
have {\sf obj} as an {\sf OUT} or {\sf INOUT} argument can be reported.
From a GraphBLAS perspective, those methods are {\em complete}.

If a GraphBLAS method, executing in nonblocking mode, returns with
any execution error from Table~\ref{Tab:ErrorValues}(b) other than
{\sf GrB\_PANIC}, it is guaranteed that no argument used as input-only
through the entire sequence has been modified.  Any output argument in
the sequence may be left in an invalid state and its use downstream in the
program flow may cause additional errors.  If a GraphBLAS method returns
with a {\sf GrB\_PANIC}, no guarantees can be made about the state of
any program data.

\begin{figure}[tb]
    \hrule
    \vspace{1\baselineskip}
    \begin{center}
        \begin{minipage}{3in}
            \begin{verbatim}
            const char *GrB_error();
            \end{verbatim}
        \end{minipage}
    \end{center}
    \caption{Signature of {\sf GrB\_error()} function.}
    \label{Fig:GrB_error}
    \hrule
\end{figure}

After a call to any GraphBLAS method, the program can retrieve additional
error information (beyond the error code returned by the method) though a
call to the function {\sf GrB\_error()}. The signature of that function is
shown in Figure~\ref{Fig:GrB_error}.  The function returns a pointer to a 
NULL-terminated string, and the contents of that string are implementation 
dependent. In particular, a null string (not a {\sf NULL} pointer) is always a valid error string.
The pointer is valid until the next call to any GraphBLAS method by the same thread.
{\sf GrB\_error()} is a thread-safe function, in the sense that multiple threads can
call it simultaneously and each will get its own error string back, referring to the
last GraphBLAS method it called.


\chapter{Objects}
\label{Chp:Objects}

The LAGraph library depends on a number of objects to represent 
graphs, vectors, and other types associated with graph algorithms in LAGraph.
Other objects are not directly associated with the input and output variables, 
but instead modify the behavior of the LAGraph functions.  These are 
related to the descriptors in the GraphBLAS.

In this chapter, we need to describe all of the objects a user of LAGraph
needs to understand.  This is also where we describe the types of LAGraph
objects and any constraints on those types.




%-----------------------------------------------------------------------------

%-----------------------------------------------------------------------------
\chapter{Functions}
\label{Chp:Functions}

The LAGraph library is composed of the following groups of functions:
\begin{itemize}

\item Context: Functions that manage the context or environment of an instance of the
LAGraph library.

\item Graph Algorithms: Functions that implement a Graph Algorithm.

\item Utilities: Functions that support implementation of Graph Algorithms or support
users of LAGraph.

\end{itemize}

We need to discuss the rules used in naming the functions and defining their argument 
lists.  

\section{Context}

LAGraph init, finalize and other functions that manage the environment of an instance of LAGraph.


\section{Graph Algorithms}

List the algorithms here.  Then have a subsection with the definition of each algorithms.

%TGM Here is an example of a method definitiion from GraphBLAS

\subsection{{\sf vxm}: Vector-matrix multiply}

Multiplies a (row) vector with a matrix on an semiring. The result is a vector.

\paragraph{\syntax}

\begin{verbatim}
        GrB_Info GrB_vxm(GrB_Vector             w,
                         const GrB_Vector       mask,
                         const GrB_BinaryOp     accum,
                         const GrB_Semiring     op,
                         const GrB_Vector       u, 
                         const GrB_Matrix       A,
                         const GrB_Descriptor   desc);
\end{verbatim}

\paragraph{Parameters}

\begin{itemize}[leftmargin=1.1in]
    \item[{\sf w}]    ({\sf INOUT}) An existing GraphBLAS vector.  On input,
    the vector provides values that may be accumulated with the result of the
    vector-matrix product.  On output, this vector holds the results of the
    operation.

    \item[{\sf mask}] ({\sf IN}) An optional ``write'' mask that controls which
    results from this operation are stored into the output vector {\sf w}. The 
    mask dimensions must match those of the vector {\sf w}. If the 
    {\sf GrB\_STRUCTURE} descriptor is {\em not} set for the mask, the domain of the
    {\sf mask} vector must be of type {\sf bool} or any of the predefined 
    ``built-in'' types in Table~\ref{Tab:PredefinedTypes}.  If the default
    mask is desired (\ie, a mask that is all {\sf true} with the dimensions of {\sf w}), 
    {\sf GrB\_NULL} should be specified.

    \item[{\sf accum}] ({\sf IN}) An optional binary operator used for accumulating
    entries into existing {\sf w} entries.
    %: ${\sf accum} = \langle \bDout({\sf accum}),\bDin1({\sf accum}),
    %\bDin2({\sf accum}), \odot \rangle$. 
    If assignment rather than accumulation is
    desired, {\sf GrB\_NULL} should be specified.

    \item[{\sf op}]   ({\sf IN}) Semiring used in the vector-matrix
    multiply.
    %: ${\sf op}=\langle \bDout({\sf op}),\bDin1({\sf op}),\bDin2({\sf op}),\oplus,\otimes,0 \rangle$.

    \item[{\sf u}]    ({\sf IN}) The GraphBLAS vector holding the values for
    the left-hand vector in the multiplication.

    \item[{\sf A}]    ({\sf IN}) The GraphBLAS matrix holding the values
    for the right-hand matrix in the multiplication.

    \item[{\sf desc}] ({\sf IN}) An optional operation descriptor. If
    a \emph{default} descriptor is desired, {\sf GrB\_NULL} should be
    specified. Non-default field/value pairs are listed as follows:  \\

    \hspace*{-2em}\begin{tabular}{lllp{2.7in}}
        Param & Field  & Value & Description \\
        \hline
        {\sf w}    & {\sf GrB\_OUTP} & {\sf GrB\_REPLACE} & Output vector {\sf w}
        is cleared (all elements removed) before the result is stored in it.\\

        {\sf mask} & {\sf GrB\_MASK} & {\sf GrB\_STRUCTURE}   & The write mask is
        constructed from the structure (pattern of stored values) of the input
        {\sf mask} vector. The stored values are not examined.\\

        {\sf mask} & {\sf GrB\_MASK} & {\sf GrB\_COMP}   & Use the
        complement of {\sf mask}. \\

        {\sf A}    & {\sf GrB\_INP1} & {\sf GrB\_TRAN}   & Use transpose of {\sf A}
        for the operation. \\
    \end{tabular}
\end{itemize}

\paragraph{Return Values}

\begin{itemize}[leftmargin=2.1in]
    \item[{\sf GrB\_SUCCESS}]         In blocking mode, the operation completed
    successfully. In non-blocking mode, this indicates that the compatibility 
    tests on dimensions and domains for the input arguments passed successfully. 
    Either way, output vector {\sf w} is ready to be used in the next method of 
    the sequence.

    \item[{\sf GrB\_PANIC}]           Unknown internal error.

    \item[{\sf GrB\_INVALID\_OBJECT}] This is returned in any execution mode 
    whenever one of the opaque GraphBLAS objects (input or output) is in an invalid 
    state caused by a previous execution error.  Call {\sf GrB\_error()} to access 
    any error messages generated by the implementation.

    \item[{\sf GrB\_OUT\_OF\_MEMORY}] Not enough memory available for the operation.

    \item[{\sf GrB\_UNINITIALIZED\_OBJECT}] One or more of the GraphBLAS objects 
    has not been initialized by a call to {\sf new} (or {\sf dup} for matrix or
    vector parameters).

    \item[{\sf GrB\_DIMENSION\_MISMATCH}] Mask, vector, and/or matrix 
    dimensions are incompatible.

    \item[{\sf GrB\_DOMAIN\_MISMATCH}]    The domains of the various vectors/matrices are
    incompatible with the corresponding domains of the semiring or
    accumulation operator, or the mask's domain is not compatible with {\sf bool}
    (in the case where {\sf desc[GrB\_MASK].GrB\_STRUCTURE} is not set).
\end{itemize}

\paragraph{Description}

{\sf GrB\_vxm} computes the vector-matrix product ${\sf w}^T = {\sf
u}^T \oplus . \otimes {\sf A}$, or, if an optional binary accumulation
operator ($\odot$) is provided, ${\sf w}^T = {\sf w}^T \odot
\left({\sf u}^T \oplus . \otimes {\sf A}\right)$ (where matrix {\sf A}
 can be optionally transposed).  Logically, this operation
occurs in three steps:
\begin{enumerate}[leftmargin=0.85in]
\item[\bf Setup] The internal vectors, matrices and mask used in the computation are formed and their domains/dimensions are tested for compatibility.
\item[\bf Compute] The indicated computations are carried out.
\item[\bf Output] The result is written into the output vector, possibly under control of a mask.
\end{enumerate}

Up to four argument vectors or matrices are used in the {\sf GrB\_vxm} operation:
\begin{enumerate}
	\item ${\sf w} = \langle \bold{D}({\sf w}),\bold{size}({\sf w}),\bold{L}({\sf w}) = \{(i,w_i) \} \rangle$
	\item ${\sf mask} = \langle \bold{D}({\sf mask}),\bold{size}({\sf mask}),\bold{L}({\sf mask}) = \{(i,m_i) \} \rangle$ (optional)
	\item ${\sf u} = \langle \bold{D}({\sf u}),\bold{size}({\sf u}),\bold{L}({\sf u}) = \{(i,u_i) \} \rangle$
	\item ${\sf A} = \langle \bold{D}({\sf A}),\bold{nrows}({\sf A}), \bold{ncols}({\sf A}),\bold{L}({\sf A}) = \{(i,j,A_{ij}) \} \rangle$
\end{enumerate}

The argument matrices, vectors, the semiring, and the accumulation operator (if provided) 
are tested for domain compatibility as follows:
\begin{enumerate}
	\item If {\sf mask} is not {\sf GrB\_NULL}, and ${\sf desc[GrB\_MASK].GrB\_STRUCTURE}$
    is not set, then $\bold{D}({\sf mask})$ must be from one of the pre-defined types of 
    Table~\ref{Tab:PredefinedTypes}.

	\item $\bold{D}({\sf u})$ must be compatible with $\bDin1({\sf op})$ of the semiring.

	\item $\bold{D}({\sf A})$ must be compatible with $\bDin2({\sf op})$ of the semiring.

	\item $\bold{D}({\sf w})$ must be compatible with $\bDout({\sf op})$ of the semiring.

	\item If {\sf accum} is not {\sf GrB\_NULL}, then $\bold{D}({\sf w})$ must be compatible with $\bDin1({\sf accum})$ and $\bDout({\sf accum})$ of the 
	accumulation operator and $\bDout({\sf op})$ of the semiring must be compatible with $\bDin2({\sf accum})$ of the accumulation operator.
\end{enumerate}
Two domains are compatible with each other if values from one domain can be cast 
to values in the other domain as per the rules of the C language.
In particular, domains from Table~\ref{Tab:PredefinedTypes} are all compatible 
with each other. A domain from a user-defined type is only compatible with itself.
If any compatibility rule above is violated, execution of {\sf GrB\_vxm} ends and 
the domain mismatch error listed above is returned.

From the argument vectors and matrices, the internal matrices and mask used in 
the computation are formed ($\leftarrow$ denotes copy):
\begin{enumerate}
	\item Vector $\vector{\widetilde{w}} \leftarrow {\sf w}$.

	\item One-dimensional mask, $\vector{\widetilde{m}}$, is computed from 
    argument {\sf mask} as follows:
	\begin{enumerate}
		\item If ${\sf mask} = {\sf GrB\_NULL}$, then $\vector{\widetilde{m}} = 
        \langle \bold{size}({\sf w}), \{i, \ \forall \ i : 0 \leq i < 
        \bold{size}({\sf w}) \} \rangle$.

		\item If {\sf mask} $\ne$ {\sf GrB\_NULL},  
        \begin{enumerate}
            \item If ${\sf desc[GrB\_MASK].GrB\_STRUCTURE}$ is set, then
            $\vector{\widetilde{m}} = 
            \langle \bold{size}({\sf mask}), \{i : i \in \bold{ind}({\sf mask}) \} \rangle$,
            \item Otherwise, $\vector{\widetilde{m}} = 
            \langle \bold{size}({\sf mask}), \{i : i \in \bold{ind}({\sf mask}) \wedge
            ({\sf bool}){\sf mask}(i) = \true \} \rangle$.
        \end{enumerate}

		\item	If ${\sf desc[GrB\_MASK].GrB\_COMP}$ is set, then 
        $\vector{\widetilde{m}} \leftarrow \neg \vector{\widetilde{m}}$.
	\end{enumerate}

	\item Vector $\vector{\widetilde{u}} \leftarrow {\sf u}$.

	\item Matrix $\matrix{\widetilde{A}} \leftarrow {\sf desc[GrB\_INP1].GrB\_TRAN} \ ? \ {\sf A}^T : {\sf A}$.
\end{enumerate}

The internal matrices and masks are checked for shape compatibility. The following 
conditions must hold:
\begin{enumerate}
	\item $\bold{size}(\vector{\widetilde{w}}) = \bold{size}(\vector{\widetilde{m}})$.

	\item $\bold{size}(\vector{\widetilde{w}}) = \bold{ncols}(\matrix{\widetilde{A}})$.

	\item $\bold{size}(\vector{\widetilde{u}}) = \bold{nrows}(\matrix{\widetilde{A}})$.
\end{enumerate}
If any compatibility rule above is violated, execution of {\sf GrB\_vxm} ends and 
the dimension mismatch error listed above is returned.

From this point forward, in {\sf GrB\_NONBLOCKING} mode, the method can 
optionally exit with {\sf GrB\_SUCCESS} return code and defer any computation 
and/or execution error codes.

We are now ready to carry out the vector-matrix multiplication and any additional 
associated operations.  We describe this in terms of two intermediate vectors:
\begin{itemize}
    \item $\vector{\widetilde{t}}$: The vector holding the product of vector
    $\vector{\widetilde{u}}^T$ and matrix $\matrix{\widetilde{A}}$.
    \item $\vector{\widetilde{z}}$: The vector holding the result after 
    application of the (optional) accumulation operator.
\end{itemize}

The intermediate vector $\vector{\widetilde{t}} = \langle
\bDout({\sf op}), \bold{ncols}(\matrix{\widetilde{A}}),
%\bold{L}(\vector{\widetilde{t}}) =
\{(j,t_j) : \bold{ind}(\vector{\widetilde{u}}) \cap
\bold{ind}(\matrix{\widetilde{A}}(:,j)) \neq \emptyset \} \rangle$
is created.  The value of each of its elements is computed by 
\[t_j = \bigoplus_{k \in \bold{ind}(\vector{\widetilde{u}}) \cap
\bold{ind}(\matrix{\widetilde{A}}(:,j))} (\vector{\widetilde{u}}(k)
\otimes \matrix{\widetilde{A}}(k,j)),\] where $\oplus$ and $\otimes$
are the additive and multiplicative operators of semiring {\sf op},
respectively.


%-----------------------------------------------------------------------------

\section{Utilities}

Import, Export, and other functions to support users and LAGraph algorithm developers.


\appendix
%
%  I include this content from the GraphBLAS spec so we have a template to work from on hand
% once we start coming out with multiple releases and come up with a good set of examples.

\chapter{Revision History}
\label{Chp:RevHistory}
%--------------------------------------------------------------

Changes in 1.3.1:
\begin{itemize}
\item (Issue 70,67) [PENDING] changes to {\sf GrB\_wait(obj)}.
\item (Issue 69) Made names/symbols containing underscores searchable in PDF.
\item Typographical change to eWiseAdd Description to be consistent in order of set intersections.
\end{itemize}


\chapter{Examples}
\label{Chp:Examples}

\pagebreak
\nolinenumbers
\section{Example: level breadth-first search (BFS) in GraphBLAS}
{\scriptsize
\lstinputlisting[language=C,numbers=left]{BFS5M.c}
}
\vfill

%
%
%\pagebreak
%\nolinenumbers
%\section{Example: level BFS in GraphBLAS using apply}
%{\scriptsize
%\lstinputlisting[language=C,numbers=left]{BFS6_apply.c}
%}
%\vfill

\pagebreak


\bibliographystyle{plain}
\bibliography{refs}

\end{document}
