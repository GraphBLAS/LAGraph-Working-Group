\hyphenation{Suite-Sparse}
\hyphenation{Graph-BLAS}
\hyphenation{Suite-Sparse-Graph-BLAS}

\newcommand{\suitesparse}{SuiteSparse\xspace}
\newcommand{\grb}{GraphBLAS\xspace}
\newcommand{\ssgrb}{SuiteSparse:GraphBLAS\xspace}
\newcommand{\gxb}{\ssgrb}
\newcommand{\lagraph}{LAGraph\xspace}
\newcommand{\pygrb}{pygraphblas\xspace}

% Define new boolean flags using etoolbox ('\newbool' is similar to '\newtoggle').
% This workaround is needed as simply putting the newcommands inside 'IfFileExists' did not do the job
% as it broke with 'Illegal parameter number in definition of \reserved@a', a symptom probably caused
% by the lack of protection (\protect). Anyways, the workaround is actually cleaner.

%\newcommand{\grbreduce}[2]{\left[{#1}_j \, {#2}(:, j) \right]}
\ifbool{ascii}{
    \newcommand{\grbm}[1]{{\ifbool{colored}{\color{brown}}{}{\mathtt{#1}}}}% matrix
    \newcommand{\grbv}[1]{{\ifbool{colored}{\color{lilac}}{}{\mathtt{#1}}}}% vector
    \newcommand{\grba}[1]{{\ifbool{colored}{\color{gray}}{}{\mathtt{#1}}}}% array
    \newcommand{\grbs}[1]{{\ifbool{colored}{\color{blue}}{}{\mathtt{#1}}}}% scalar

    \newcommand{\grbstr}[1]{{\{#1\}}}
    \newcommand{\grbmask}[1]{<\! #1 \!>}
    \newcommand{\grbmaskreplace}[1]{<\!<\! #1 \!>\!>}
    \newcommand{\grbneg}{\texttt{!}}
    \newcommand{\grbassign}{\mathrel{\texttt{<-}}}
    \newcommand{\grbf}[2]{\texttt{#1}(#2)}
    \newcommand{\grbreduce}[4]{[ {#1 #3} ]} % omit the indices
    \newcommand{\grbt}{\texttt{'}} % transpose
    \newcommand{\grbdiv}{\grbbinaryop{DIV}}
    \newcommand{\grbminus}{\grbbinaryop{MINUS}}
    \newcommand{\grbaccumeq}[1]{\mathbin{\texttt{\ensuremath{\ifstrempty{#1}{+}{#1}=}}}}

    \newcommand{\grbplus}{\grbbinaryop{+}}
    \newcommand{\grbtimes}{\grbbinaryop{\times}}
    \newcommand{\grbapply}{\grbbinaryop{\odot}}

    \newcommand{\grbfrac}[2]{(#1)/(#2)}

    \newcommand{\grbbool}{\mathtt{bool}} % booleans
    \newcommand{\grbuint}{\mathtt{uint}} % unsigned integers
    \newcommand{\grbint}{\mathtt{int}}   % integers
    \newcommand{\grbfloat}{\mathtt{fp}}  % floats (?)

    \newcommand{\grbplaceholder}[1]{\mathsf{#1}}

    \newcommand{\grbscalartype}[2]{\mathtt{#1#2()}}
    \newcommand{\grbvectortype}[3]{\mathtt{#1#2(#3)}}
    \newcommand{\grbmatrixtype}[4]{\mathtt{#1#2(#3, #4)}}

    \newcommand{\grbnewscalar}[3]{\mathtt{#1 = \grbscalartype{#2}{#3}}}
    \newcommand{\grbnewvector}[4]{\mathtt{#1 = \grbvectortype{#2}{#3}{#4}}}
    \newcommand{\grbnewmatrix}[5]{\mathtt{#1 = \grbmatrixtype{#2}{#3}{#4}{#5}}}

    \newcommand{\grbalpha}{\mathtt{alpha}}
    \newcommand{\grboperator}[1]{\mathtt{#1}}

    \newcommand{\grbrange}[2]{#1:#2}
    \newcommand{\grbdontcare}{\_}
}{ % LaTeX mode
    \newcommand{\grbm}[1]{{\ifbool{colored}{\color{brown}}{}{\mathbf{#1}}}}% matrix
    \newcommand{\grbv}[1]{{\ifbool{colored}{\color{lilac}}{}{\mathbf{#1}}}}% vector
    \newcommand{\grba}[1]{{\ifbool{colored}{\color{gray}}{}{\mathit{#1}}}}% array
    \newcommand{\grbs}[1]{{\ifbool{colored}{\color{blue}}{}{\mathit{#1}}}}% scalar

    \newcommand{\grbstr}[1]{{\{#1\}}}
    \newcommand{\grbmask}[1]{\langle #1 \rangle}
    \newcommand{\grbmaskreplace}[1]{\langle\!\langle #1 \rangle\!\rangle}
    \newcommand{\grbneg}{\neg}

    % use the \mapsfrom symbol extracted from the stix package as suggested in https://tex.stackexchange.com/a/331899/71109
    \DeclareFontEncoding{LS1}{}{}
    \DeclareFontSubstitution{LS1}{stix}{m}{n}
    \DeclareSymbolFont{arrows1}{LS1}{stixsf}{m}{n}
    \DeclareMathSymbol{\mapsfrom}{\mathrel}{arrows1}{"AB}
    \newcommand{\grbassign}{\mapsfrom}

    \newcommand{\grbf}[2]{\grboperation{#1}{#2}}
    \newcommand{\grbreduce}[4]{[ {#1}_{#2}\, #3(#4) ]}
    \newcommand{\grbt}{^{\top}} % transpose
    \newcommand{\grbdiv}{\grbbinaryop{\oslash}}
    \newcommand{\grbminus}{\grbbinaryop{\ominus}}
    \newcommand{\grbaccumeq}[1]{\mathbin{\ensuremath{\ifstrempty{#1}{\odot}{#1}\!\!=}}}

    \newcommand{\grbplus}{\oplus}
    \newcommand{\grbtimes}{\otimes}
    \newcommand{\grbapply}{\odot}
    
    \newcommand{\grbfrac}[2]{\frac{#1}{#2}}

    \newcommand{\grbbool}{\mathbb{B}}  % booleans
    \newcommand{\grbuint}{\mathbb{N}}  % unsigned integers
    \newcommand{\grbint}{\mathbb{Z}}   % integers
    \newcommand{\grbfloat}{\mathbb{Q}} % floats (?)
    
    \newcommand{\grbplaceholder}[1]{\mathsf{#1}}

    \newcommand{\grbscalartype}[2]{#1_{#2}}
    \newcommand{\grbvectortype}[3]{#1_{#2}^{#3}}
    \newcommand{\grbmatrixtype}[4]{#1_{#2}^{#3 \times #4}}

    \newcommand{\grbnewscalar}[3]{\text{let: } #1 \in \grbscalartype{#2}{#3}}
    \newcommand{\grbnewvector}[4]{\text{let: } #1 \in \grbvectortype{#2}{#3}{#4}}
    \newcommand{\grbnewmatrix}[5]{\text{let: } #1 \in \grbmatrixtype{#2}{#3}{#4}{#5}}

    \newcommand{\grbalpha}{\alpha}
    \newcommand{\grboperator}[1]{\mathsf{#1}}

    \newcommand{\grbrange}[2]{#1 \! : \! #2}
    \newcommand{\grbdontcare}{\textvisiblespace}
}


% do not lange/rangle for tuples as it is already used for masks
% do not use grbtuple for the time being
%\newcommand{\grbtuple}[1]{( #1 )}


% trying to avoid too much syntax (e.g. using wedge/vee symbols for LAND/LOR)
%\newcommand{\grblorland}{\lor\!.\!\land}

\newcommand{\grbsemiringops}[2]{\mathbin{\grboperator{#1.#2}}}
\newcommand{\grbplustimes}{\grbsemiringops{\grbplus}{\grbtimes}}

\newcommand{\grbanypair}{\grbsemiringops{any}{pair}}
\newcommand{\grbanyfirst}{\grbsemiringops{any}{first}}
\newcommand{\grbanysecond}{\grbsemiringops{any}{second}}
\newcommand{\grblorland}{\grbsemiringops{lor}{land}}
\newcommand{\grbminplus}{\grbsemiringops{min}{plus}}
\newcommand{\grbmaxplus}{\grbsemiringops{max}{plus}}
\newcommand{\grbmaxfirst}{\grbsemiringops{max}{first}}
\newcommand{\grbminfirst}{\grbsemiringops{min}{first}}
\newcommand{\grbminsecond}{\grbsemiringops{min}{second}}
\newcommand{\grbmaxsecond}{\grbsemiringops{max}{second}}
\newcommand{\grbsecondmin}{\grbsemiringops{second}{min}}
\newcommand{\grbsecondmax}{\grbsemiringops{second}{max}}
\newcommand{\grbarithmeticplustimes}{\grbsemiringops{plus}{times}} % not necessary because this is the default

\newcommand{\grbbinaryop}[1]{\mathop{\grboperator{#1}}}
\newcommand{\grbany}{\grbbinaryop{any}}
\newcommand{\grbpair}{\grbbinaryop{pair}}
\newcommand{\grbland}{\grbbinaryop{land}}
\newcommand{\grblor}{\grbbinaryop{lor}}
\newcommand{\grbmin}{\grbbinaryop{min}}
\newcommand{\grbmax}{\grbbinaryop{max}}
\newcommand{\grbfirst}{\grbbinaryop{first}}
\newcommand{\grbsecond}{\grbbinaryop{second}}
\newcommand{\grbfirsti}{\grbbinaryop{firsti}}
\newcommand{\grbsecondi}{\grbbinaryop{secondi}}
\newcommand{\grbfirstii}{\grbbinaryop{firsti1}}
\newcommand{\grbsecondii}{\grbbinaryop{secondi1}}
\newcommand{\grbarithmeticplus}{\grbbinaryop{plus}} % usually not necessary because this is the default addition
\newcommand{\grbarithmetictimes}{\grbbinaryop{times}} % usually not necessary because this is the default multiplication
\newcommand{\grbisne}{\grbbinaryop{isne}}

% boolean values
\newcommand{\grbbooleanvalue}[1]{\mathtt{#1}}
\newcommand{\grbtrue}{\grbbooleanvalue{TRUE}}
\newcommand{\grbfalse}{\grbbooleanvalue{FALSE}}
\newcommand{\grbstring}{\textrm{String}}
\newcommand{\grbdate}{\textrm{Date}}

% cardinality / count
\newcommand{\grbcnt}[1]{| #1 |}

\newcommand{\grbeWiseAdd}[1]{\underset{\raisebox{.4ex}{$\scriptscriptstyle\cup$}}{#1}}
\newcommand{\grbeWiseMult}[1]{\underset{\raisebox{.4ex}{$\scriptscriptstyle\cap$}}{#1}}

% C<M> += A +.* B
%\newcommand{\grbmult}[6]{#1{#2} \ #3\!= #5 #4 #6}

% stock matrices
\newcommand{\grbmA}{\grbm{A}}
\newcommand{\grbmB}{\grbm{B}}
\newcommand{\grbmC}{\grbm{C}}
\newcommand{\grbmM}{\grbm{M}}
% stock vectors
\newcommand{\grbvu}{\grbv{u}}
\newcommand{\grbvv}{\grbv{v}}
\newcommand{\grbvw}{\grbv{w}}
\newcommand{\grbvm}{\grbv{m}}
% stock scalars
\newcommand{\grbsi}{\grbs{i}}
\newcommand{\grbsj}{\grbs{j}}
\newcommand{\grbsn}{\grbs{n}}
\newcommand{\grbss}{\grbs{s}}
% stock arrays
\newcommand{\grbaI}{\grba{I}}
\newcommand{\grbaJ}{\grba{J}}
\newcommand{\grbaX}{\grba{X}}

% operations
\newcommand{\grboperationnoarg}[1]{\mathit{#1}}
\newcommand{\grboperation}[2]{\grboperationnoarg{#1}(#2)}
\newcommand{\grbnrows}[1]{\grboperation{nrows}{#1}}
\newcommand{\grbncols}[1]{\grboperation{ncols}{#1}}
\newcommand{\grbnvals}[1]{\grboperation{nvals}{#1}}
\newcommand{\grbclear}[1]{\grboperation{clear}{#1}}
\newcommand{\grbdiag}[1]{\grboperation{diag}{#1}}
\newcommand{\grbselect}[1]{\grboperation{select}{#1}}
\newcommand{\grbkron}[1]{\grboperation{kron}{#1}}
\newcommand{\grbtril}[1]{\grboperation{tril}{#1}}
\newcommand{\grbtriu}[1]{\grboperation{triu}{#1}}
\newcommand{\grbondiag}[1]{\grboperation{ondiag}{#1}}
\newcommand{\grboffdiag}[1]{\grboperation{offdiag}{#1}}

% bfs
\newcommand{\grbFrontier}{\grbm{Frontier}}
\newcommand{\grbNext}{\grbm{Next}}
\newcommand{\grbSeen}{\grbm{Seen}}

\newcommand{\grbsource}{\grbs{s}}
\newcommand{\grbfrontier}{\grbv{frontier}}
\newcommand{\grbnext}{\grbv{next}}
\newcommand{\grbseen}{\grbv{seen}}
\newcommand{\grblevel}{\grbs{level}}

\newcommand{\grbfrontieri}{\grbv{frontier1}}
\newcommand{\grbnexti}{\grbv{next1}}
\newcommand{\grbseeni}{\grbv{seen1}}

\newcommand{\grbfrontierii}{\grbv{frontier2}}
\newcommand{\grbnextii}{\grbv{next2}}
\newcommand{\grbseenii}{\grbv{seen2}}

\newcommand{\bfsfrontier}{\grbv{frontier}}
\newcommand{\bfsnext}{\grbv{next}}
\newcommand{\bfsseen}{\grbv{seen}}
