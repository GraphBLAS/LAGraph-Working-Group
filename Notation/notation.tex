\documentclass[11pt]{article}
% \textwidth=6in
% \textheight=9in
% \oddsidemargin=0.25in
% \evensidemargin=0.25in
% \marginparwidth=0in
% \marginparsep=0in
% \topmargin=0in
% \headheight=0in
% \headsep=0in
% \topskip=0in
\usepackage{amssymb}
\usepackage{hyperref}
\usepackage{lscape}
\usepackage{longtable}
\usepackage{graphics}
\usepackage[pdftex]{graphicx}
\usepackage{color}
\usepackage{rotating}
\newcommand{\hilight}[1]{\colorbox{yellow}{#1}}

\newenvironment{packed_itemize}{
\begin{itemize}
  \setlength{\itemsep}{1pt}
  \setlength{\parskip}{0pt}
  \setlength{\parsep}{0pt}
}{\end{itemize}}

\title{Options for a consistent GraphBLAS notation}
\author{Tim Davis}
\date{\today}
\begin{document}

\maketitle

\section{Creating matrices}

\begin{itemize}
\item[] in the C API: \verb'GrB_Matrix_new (&A, m, n, GrB_INT64)'
\item[] in ASCII: \verb'A = int16(m,n)' (I'm open to suggestions for this)
\item[] in LaTeX: $\mbox{let: } {\bf A} \in \mathbb{Z}^{m \times n}_{16}$
\end{itemize}

\section{Mask computation}

In the following table the ``mask'' column is the value of the
mask after applying the value/structural option and optionally
complementing the mask.  If no mask is present (and not complemented)
then mask is 1.  If no mask is present (yet complemented) then the mask is
always 0; this has no effect unless the Replace option is also used.

The repl column in the table is ``yes'' if Replace is enabled, ``no'' otherwise.
The accum column in the table is ``yes'' if the accum is 
present, ``no'' otherwise.
The ${\bf C}$ column is listed as $c_{ij}$ if that entry is present,
and a dash otherwise, likewise for the ${\bf A}$ column.

\begin{table}
{\small
\begin{tabular}{lllll|l}
\hline
repl & accum & ${\bf C}$ & ${\bf A}$ & mask & action taken by ${\bf C \langle M \rangle = C \odot A}$ \\
\hline
\hline
       &    &          &           &      & \verb'C<M>=A' and \verb'C<!M>=A' \\
\hline
    -  &-   & $c_{ij}$ & $a_{ij}$  & 1    &  $c_{ij} = a_{ij}$, update \\
    -  &-   &  -       & $a_{ij}$  & 1    &  $c_{ij} = a_{ij}$, insert \\
    -  &-   & $c_{ij}$ &  -        & 1    &  delete $c_{ij}$ because $a_{ij}$ not present \\
    -  &-   &  -       &  -        & 1    &   \\
\hline
    -  &-   & $c_{ij}$ & $a_{ij}$  & 0    &   \\
    -  &-   &  -       & $a_{ij}$  & 0    &   \\
    -  &-   & $c_{ij}$ &  -        & 0    &   \\
    -  &-   &  -       &  -        & 0    &   \\
\hline
\hline
       &    &          &           &      & \verb'C<M,repl>=A' and \verb'C<!M,repl>=A' \\
\hline
    yes&-   & $c_{ij}$ & $a_{ij}$  & 1    &  $c_{ij} = a_{ij}$, update \\
    yes&-   &  -       & $a_{ij}$  & 1    &  $c_{ij} = a_{ij}$, insert \\
    yes&-   & $c_{ij}$ &  -        & 1    &  delete $c_{ij}$ because $a_{ij}$ not present \\
    yes&-   &  -       &  -        & 1    &   \\
\hline
    yes&-   & $c_{ij}$ & $a_{ij}$  & 0    &  delete $c_{ij}$  (because of \verb'GrB_REPLACE') \\
    yes&-   &  -       & $a_{ij}$  & 0    &   \\
    yes&-   & $c_{ij}$ &  -        & 0    &  delete $c_{ij}$  (because of \verb'GrB_REPLACE') \\
    yes&-   &  -       &  -        & 0    &   \\
\hline
\hline
       &    &          &           &      & \verb'C<M>+=A' and \verb'C<!M>+=A' \\
\hline
    -  &yes & $c_{ij}$ & $a_{ij}$  & 1    &  $c_{ij} = c_{ij} \odot a_{ij}$, apply accumulator \\
    -  &yes &  -       & $a_{ij}$  & 1    &  $c_{ij} = a_{ij}$, insert \\
    -  &yes & $c_{ij}$ &  -        & 1    &   \\
    -  &yes &  -       &  -        & 1    &   \\
\hline
    -  &yes & $c_{ij}$ & $a_{ij}$  & 0    &   \\
    -  &yes &  -       & $a_{ij}$  & 0    &   \\
    -  &yes & $c_{ij}$ &  -        & 0    &   \\
    -  &yes &  -       &  -        & 0    &   \\
\hline
\hline
       &    &          &           &      & \verb'C<M,repl>+=A' and \verb'C<!M,repl>+=A' \\
\hline
    yes&yes & $c_{ij}$ & $a_{ij}$  & 1    &  $c_{ij} = c_{ij} \odot a_{ij}$, apply accumulator \\
    yes&yes &  -       & $a_{ij}$  & 1    &  $c_{ij} = a_{ij}$, insert \\
    yes&yes & $c_{ij}$ &  -        & 1    &   \\
    yes&yes &  -       &  -        & 1    &   \\
\hline

\hline
    yes&yes & $c_{ij}$ & $a_{ij}$  & 0    &  delete $c_{ij}$  (because of \verb'GrB_REPLACE') \\
    yes&yes &  -       & $a_{ij}$  & 0    &   \\
    yes&yes & $c_{ij}$ &  -        & 0    &  delete $c_{ij}$  (because of \verb'GrB_REPLACE') \\
    yes&yes &  -       &  -        & 0    &   \\
\hline
\end{tabular}
}
\caption{Results of the mask/accumulator phase \label{tab:maskaccum}}
\end{table}

\newpage
\section{For Mask/Accum/Replace: there are 32 variants}

All these options can affect each other so there are 32 variants total,
except that when no mask is present, the value/structural option makes
no difference.

\begin{itemize}
\item the mask ${\bf M}$ can be present, or not present.
\item the mask can be complemented, or not (even when no mask is present)
\item the mask can be valued or structural
\item the replace option can be enabled, or not.
\item the accumulator can be present, or not.
\end{itemize}

\begin{itemize}
\item plain mask, with accum:

    \begin{itemize}
    \item[] in ASCII: \verb'C<M>+=A'
    \item[] in LaTex: $\bf C \langle M \rangle \odot\!\!= A $
    \end{itemize}

\item negated mask, with accum.  Several options for LaTeX.
I like the $\overline{\bf M}$ since it is more compact, but it
has problems when $\bf M$ is also structural (see below).

    \begin{itemize}
    \item[] in ASCII: \verb'C<!M>+=A'
    \item[] in LaTex: $\bf C \langle \neg M \rangle \odot\!\!= A$
    \item[] in LaTex:  $\bf C \langle \overline{M} \rangle \odot\!\!= A$
    \end{itemize}

\item structural mask, not complemented:
using $s({\bf M})$ where $s$ stands for structural.
This feels wordy to me.  Not elegant.
The letter $s$ could be confused for a vector or scalar.

    \begin{itemize}
    \item[] in ASCII: \verb'C<s(M)>+=A'
    \item[] in LaTex:  ${\bf C} \langle s({\bf M}) \rangle \odot\!\!= {\bf A}$
    \end{itemize}

\item structural mask alternatives.  The letter $s$ could be confused
for a scalar or vector, so a Greek letter could be used instead,
such as sigma ($\sigma$).  But this is also wordy.

    \begin{itemize}
    \item[] in ASCII: \verb'C<s(M)>+=A'
    \item[] in LaTex:  ${\bf C} \langle \sigma({\bf M}) \rangle \odot\!\!= {\bf A}$
    \end{itemize}

\item structural mask alternative.  The notation 
$s({\bf M})$ is a little wordy.  The Mask cannot be transposed in a
single call to \verb'GrB_anything' so a superscript would work.  But a
subscript might be better, because someone might want to indicate a
transposed mask (which would have to be done as a separate call to
\verb'GrB_transpose').  This doesn't look good in ASCII however:

    \begin{itemize}
    \item[] in ASCII: \verb'C<M_s>+=A'
    \item[] in LaTex:  ${\bf C} \langle {\bf M}_s \rangle \odot\!\!= {\bf A}$
    \end{itemize}

\item structural mask: why not use set notation?  The structure of ${\bf M}$
is a set of the entries of ${\bf M}$.  In a sense, selecting the
structural option in the descriptor converts ${\bf M}$
into a set.  So how about set notation, with curly brackets?
We don't use curly brackets anywhere else in GraphBLAS, so this is unambigous.
I like this the best.

    \begin{itemize}
    \item[] in ASCII: \verb'C<{M}>+=A'
    \item[] in LaTex:  ${\bf C} \langle \{{\bf M}\} \rangle \odot\!\!= {\bf A}$
    \end{itemize}

\item complemented structural mask, for $s({\bf M})$:

    \begin{itemize}
    \item[] in ASCII: \verb'C<!s(M)>+=A'
    \item[] in LaTex:  ${\bf C} \langle \neg s({\bf M}) \rangle \odot\!\!= {\bf A}$
    \end{itemize}

\item  complemented structural mask, for the curly bracket option:
It is important to note that the structure of ${\bf M}$ is taken
first, and then complemented.  Not the other way around.
I like this option the best (with $\neg$).

    \begin{itemize}
    \item[] ASCII: \verb'C<!{M}>+=A'
    \item[] LaTeX: ${\bf C} \langle \neg \{{\bf M}\} \rangle \odot\!\!= {\bf A}$
    \item[] LaTeX:
    ${\bf C} \langle \overline{ \{{\bf M}\}} \rangle \odot\!\!= {\bf A}$
    ouch.  That is a long overline, and could be confused with
    ${\bf C} \langle  \{\overline{\bf M}\} \rangle \odot\!\!= {\bf A}$
    \end{itemize}

\end{itemize}

\newpage

\section{Replace option}

The examples consider all options except for Replace.
Here are some ideas; I am not really happy with any of them.

\begin{itemize}

\item simple case: a mask with replace

    \begin{itemize}
    \item[] in ASCII: \verb'C<M,replace>+=A'
    \item[] in LaTex: $\bf C \langle M, \mbox{replace} \rangle \odot\!\!= A $
    \end{itemize}

\item complemented structural mask with replace

    \begin{itemize}
    \item[] ASCII: \verb'C<!{M},replace>+=A'
    \item[] LaTeX: ${\bf C} \langle \neg \{{\bf M}\}, \mbox{replace}
    \rangle \odot\!\!= {\bf A}$
    \end{itemize}

\item no mask, yet complemented, with replace.  This does actual work that
does not depend on the input matrix $\bf A$ or input scalar.

    \begin{itemize}
    \item[] ASCII: \verb'C<!,replace>=A'
    \item[] LaTeX: ${\bf C} \langle \neg, \mbox{replace}
    \rangle \odot\!\!= {\bf A}$
    \end{itemize}

\end{itemize}

\section{Different accum operators}

The accum operator can be any function, such as PLUS, TIMES, FIRST, MIN, MAX,
ANY, logical OR, logical AND etc.  I propose using using symbols from C/C++ if 
available, or spelling out the function otherwise.
Assuming no mask/replace/complement, in ASCII:

    \begin{itemize}
    \item[] \verb'C +=A ' plus
    \item[] \verb'C *= A ' times
    \item[] \verb'C &&= A ' logical AND
    \item[] \verb'C &= A ' bitwise AND
    \item[] \verb'C ||= A ' logical OR
    \item[] \verb'C |= A ' bitwise OR
    \item[] \verb'C ^= A ' logical XOR, but see the XOR in LaTeX below
    \item[] \verb'C /= A ' divide
    \item[] \verb'C max= A ' maximum
    \item[] \verb'C any= A ', this is Tim's \verb'GxB_ANY_*' operator.
    \item[] \verb'C first= A '
    \item[] and so on
    \end{itemize}

In LaTeX, it is very similar.  Note the spacing:

    \begin{itemize}
    \item[] $\bf C \,+\!\!= A $, plus
    \item[] Latex source of the above: \verb'$\bf C \,+\!\!= A$'
    \item[] $\bf C \,*\!\!= A $, times
    \item[] $\bf C \,\wedge\!\!= A $, logical AND
    \item[] $\bf C \,\&\!\!= A $, bitwise AND
    \item[] $\bf C \,\vee\!\!= A $, logical OR
    \item[] $\bf C \,|\!\!= A $, bitwise OR
    \item[] $\bf C \,\oplus\!\!= A $, logical XOR?  but this conflicts with
        the use of $\oplus$ to denote a generic additive operator.
        So perhaps logical XOR should be written
        $\bf C \,\mbox{xor}\!\!= A $.
    \item[] $\bf C \,/\!\!= A $, divide, or perhaps
        $\bf C \,\div\!\!= A $
    \item[] $\bf C \,\mbox{max}\!\!= A $, maximum
    \item[] $\bf C \,\mbox{any}\!\!= A $, ANY
    \item[] $\bf C \,\mbox{first}\!\!= A $, FIRST
    \item[] and so on
    \end{itemize}

\end{document}

